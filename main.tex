\documentclass[a4paper, 12pt]{article}

\usepackage[utf8]{inputenc}
\usepackage{graphicx}
\usepackage{geometry}
\usepackage{amsmath}
\usepackage{amssymb}
\usepackage{float} % Om te bepalen waar figuren juist geplaatst worden.
\usepackage{listings}

% \geometry{tmargin=3cm, bmargin=2.2cm, lmar2.2cm, rmargin=2cm}

% Geen geïndenteerde paragrafen, da's maar lelijk.
\setlength{\parindent}{0pt}
% Wel spatie tussen paragrafen please.
\setlength{\parskip}{1em}

\begin{document}

\begin{titlepage}
    \newpage
    \thispagestyle{empty}
    \frenchspacing
    \hspace{-0.2cm}
    \includegraphics[height=4.2cm]{fig/sedes}
    \hspace{0.2cm}
    \rule{0.5pt}{4.2cm}
    \hspace{0.2cm}
    \begin{minipage}[b]{8cm}
        \Large{Katholieke\newline Universiteit\newline Leuven}\smallskip\newline
%         \large{}\smallskip\newline
%         \textbf{Departement Wiskunde}\newline
%         Afdeling Statistiek
    \end{minipage}
    \hspace{\stretch{1}}
    \vspace*{3.2cm}\vfill
    \begin{center}
        \begin{minipage}[t]{\textwidth}
            \begin{center}
            	\LARGE{\rm{Statistische Modellen \& Data-analyse}}\\[5mm]
                \LARGE{\rm{\textbf{Practicum 2}}}
            \end{center}
        \end{minipage}
    \end{center}
    \vfill
    \hfill\makebox[8.5cm][l]{%
        \vbox to 7cm{\vfill\noindent
        	\large{{\rm\textbf{Tomas Fiers} -- r0380267}\\[2mm]
                   {\rm Juni 2017}}
        }
    }
\end{titlepage}

\section{Linear regression}
\subsection*{}
\subsection*{Explorative analysis \& outlier removal}
boxplot of noise, identify outliers
% \begin{figure}
% \begin{center}
% \includegraphics[width=1.0\textwidth]{jacobi_convergentie}
% \end{center}
% \caption{Convergentiesnelheid van de Jacobi-methode voor de matrix \texttt{mat1}.}
% \label{Jacobi_convergentie}
% \end{figure}
\subsection*{Variable transformation}
\subsection*{Variable selection}
\subsection*{Linear model}
\subsection*{Gauss-Markov conditions}

\section{Classification}
\subsection*{Variable selection}
\subsection*{Logistic model}
\subsection*{Interpretation}
\subsection*{Apparent error rate}
\subsection*{Error rate on test set}
\subsection*{Linear discriminant analysis}
\subsection*{Quadratic discriminant analysis}
\subsection*{Comparison}

% \begin{figure}
% \begin{center}
% \includegraphics[width=1.0\textwidth]{jacobi_convergentie}
% \end{center}
% \caption{Convergentiesnelheid van de Jacobi-methode voor de matrix \texttt{mat1}.}
% \label{Jacobi_convergentie}
% \end{figure}

% \begin{table}[h!]
% \centering
% \begin{tabular}{r | l l l l l l l l }
% Eigenwaarde & 1    &  2  \\
% \hline
% Bijdrage    & 0.55 &  0.21  \\
% Cumulatief  & 0.55 &  0.76  \\
% \end{tabular}
% \caption{Relatieve bijdrage van elke eigenwaarde van de correlatiematrix aan de totale variantie van de data.}
% \label{PCA_eigenwaarden_bijdragen}
% \end{table}

\clearpage
\appendix

% \section{Code}
% \subsection*{1.R}
% \lstinputlisting[basicstyle=\ttfamily\scriptsize,language=R,breaklines=true]{1.R}

\end{document}
